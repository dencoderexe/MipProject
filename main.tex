% Metódy inžinierskej práce

\documentclass[10pt,twoside,slovak,a4paper]{article}

\usepackage[slovak]{babel}
%\usepackage[T1]{fontenc}
\usepackage[IL2]{fontenc} % lepšia sadzba písmena Ľ než v T1
\usepackage[utf8]{inputenc}
\usepackage{graphicx}
\usepackage{url} % príkaz \url na formátovanie URL
\usepackage{hyperref} % odkazy v texte budú aktívne (pri niektorých triedach dokumentov spôsobuje posun textu)

\usepackage{cite}
%\usepackage{times}

\pagestyle{headings}

\title{Praktická aplikácia regulárnych výrazov. Problémy a ťažkosti pri ich písaní a používaní.\thanks{Semestrálny projekt v predmete Metódy inžinierskej práce, ak. rok 2023/24, vedenie: Denis Danilov}} % meno a priezvisko vyučujúceho na cvičeniach

\author{Denis Danilov\\[2pt]
	{\small Slovenská technická univerzita v Bratislave}\\
	{\small Fakulta informatiky a informačných technológií}\\
	{\small \texttt{...@stuba.sk}}
	}

\date{\small 5. október 2023} % upravte



\begin{document}

\maketitle

\begin{abstract}
	Regulárne výrazy, často známe ako regex alebo regexp, je formálny jazyk, ktorý sa používa na priraďovanie textových vzorov. Regulárne výrazy sú široko používané a obľúbené, preto programátori vyvinuli nástroje na implementáciu ich tvorby, overovania a používania pre rôzne programovacie jazyky, či už priamo alebo prostredníctvom knižníc. Regulárne výrazy sú podporované jazykmi Python, Java, JavaScript, C, C++, C\# a PHP. Údaje naznačujú, že regulárne výrazy sa používajú vo viac ako tretine aplikácií Java a Python.
	Cieľom tejto práce je preskúmať, ako sa regulárne výrazy často používajú v praxi, identifikovať problémy a ťažkosti, s ktorými sa programátori stretávajú pri písaní a používaní regulárnych výrazov, a spôsoby, ako zlepšiť ich výkon.
\end{abstract}



\section{Úvod}



\section{Záver} \label{zaver} % prípadne iný variant názvu



%\acknowledgement{Ak niekomu chcete poďakovať\ldots}


% týmto sa generuje zoznam literatúry z obsahu súboru literatura.bib podľa toho, na čo sa v článku odkazujete
\bibliography{literatura}
\bibliographystyle{plain} % prípadne alpha, abbrv alebo hociktorý iný
\end{document}